\documentclass{report}





\usepackage{amsmath,amssymb,geometry}


\newcommand{\eqn}[2]{
\begin{equation}\begin{split}
#1
\label{#2}
\end{split}\end{equation}
}


\Large
\begin{document}








Two JavaFX rectangles with constructor arguments $(X_1,Y_1,W_1,H_1)$ and $(X_2,Y_2,W_2,H_2)$ are given. Two points $(A_1,B_1)$ and $(A_2,B_2)$ are chosen, one for each rectangle. If both the rectangles rotate as much as $\theta$, the new coordinates are
\eqn{
&
X_1^\text{rot}=
X_1+\frac{W_1}{2}
+\left(A_1-X_1-\frac{W_1}{2}\right)\cos\theta
-\left(B_1-Y_1-\frac{H_1}{2}\right)\sin\theta,
\\&
Y_1^\text{rot}=
Y_1+\frac{H_1}{2}
+\left(A_1-X_1-\frac{W_1}{2}\right)\sin\theta
+\left(B_1-Y_1-\frac{H_1}{2}\right)\cos\theta,
}{}
and
\eqn{
&
X_2^\text{rot}=
X_2+\frac{W_2}{2}
+\left(A_2-X_2-\frac{W_2}{2}\right)\cos\theta
-\left(B_2-Y_2-\frac{H_2}{2}\right)\sin\theta,
\\&
Y_2^\text{rot}=
Y_2+\frac{H_2}{2}
+\left(A_2-X_2-\frac{W_2}{2}\right)\sin\theta
+\left(B_2-Y_2-\frac{H_2}{2}\right)\cos\theta.
}{}
We enforce $X_1^\text{rot}=X_2^\text{rot}$ and $Y_1^\text{rot}=Y_2^\text{rot}$. We know that $A_i-X_i-\frac{W_i}{2}=a_i\frac{W_i}{2}$ and $B_i-Y_i-\frac{H_i}{2}=b_i\frac{H_i}{2}$. These values are given in the following table for the eight corner points. Hence,
\eqn{
&
X_2+\frac{W_2}{2}
+\frac{a_2W_2}{2}\cos\theta
-\frac{b_2H_2}{2}\sin\theta
=
X_1+\frac{W_1}{2}
+\frac{a_1W_1}{2}\cos\theta
-\frac{b_1H_1}{2}\sin\theta
\\&
Y_2+\frac{H_2}{2}
+\frac{a_2W_2}{2}\sin\theta
+\frac{b_2H_2}{2}\cos\theta
=
Y_1+\frac{H_1}{2}
+\frac{a_1W_1}{2}\sin\theta
+\frac{b_1H_1}{2}\cos\theta
}{}
or
\eqn{
&
X_2
=
X_1+\frac{W_1-W_2}{2}
+\frac{a_1W_1-a_2W_2}{2}\cos\theta
-\frac{b_1H_1-b_2H_2}{2}\sin\theta
\\&
Y_2
=
Y_1+\frac{H_1-H_2}{2}
+\frac{a_1W_1-a_2W_2}{2}\sin\theta
+\frac{b_1H_1-b_2H_2}{2}\cos\theta.
}{}

\begin{table}[h]
\centering
\begin{tabular}{|c|c|c|}
\hline
index&$a_1$&$b_1$\\\hline
0&-1&-1\\\hline
1&-1&0\\\hline
2&-1&1\\\hline
3&0&1\\\hline
4&1&1\\\hline
5&1&0\\\hline
6&1&-1\\\hline
7&0&-1\\\hline
\end{tabular}
\end{table}

We can assume that $a_1=a_2=a$ and $b_1=b_2=b$. Hence, by defining $H_1-H_2=\delta_\text{H}$ and $W_1-W_2=\delta_\text{W}$ we have
\eqn{
&
X_2=X_1+\frac{\delta_\text{W}}{2}(1+a\cos\theta)-\frac{b\delta_\text{H}}{2}\sin\theta,
\\&
Y_2=Y_1+\frac{a\delta_\text{W}}{2}\sin\theta+\frac{\delta_\text{H}}{2}(1+b\cos\theta).
}{}

\newpage

On dynamicDragRectangle:

\begin{enumerate}
\item
onMousePressed
\begin{enumerate}
\item
On Ctrl button up:
\begin{enumerate}
\item
If pressed mouse on no shape (mouse press location contained by no shape), unselect all shapes and set to drawing dynamicDragRectangle.
\item
If pressed mouse on an unselected shape, unselect all and select the highest-layer shape containing the mouse press location.
\item
If pressed mouse on a selected shape, do nothing.
\end{enumerate}
\end{enumerate}
\item
onMouseDragged
\begin{enumerate}
\item
On Ctrl button up or down:
\begin{enumerate}
\item
I suspect we only need to move the selected objects.
\end{enumerate}
\end{enumerate}
\end{enumerate}





\section{Resizing}


The resizing transformation can be generally described as two independent resizing along two perpendicular vectors. WLOG, we can assume these vectors to be $v_1=[\cos \theta,\sin \theta]$ and $v_2=[-\sin \theta,\cos \theta]$. They resize, so that the new vectors are $v_1^\text{new}=[s_x\cos \theta,s_x\sin \theta]$ and $v_2^\text{new}=[-s_y\sin \theta,s_y\cos \theta]$. We wish to find out what happens to the whole space.

Fortunately, such a transformation linear and invertible. This means that we can model whole the transformation process by a matrix. We assume then, that the transformation is a result of three consecutive transformations:

1- The plane is rotated as much as $-\theta$.

2- The plane is scaled as much as $s_x$ and $s_y$ along the x- and y- axes, resp.

3- The plane is rotated as much as $\theta$.

The resizing transformation matrix is then
\eqn{
R&=\begin{bmatrix}
\cos \theta&-\sin\theta\\
\sin \theta&\cos\theta
\end{bmatrix}
\begin{bmatrix}
s_x&0\\0&s_y
\end{bmatrix}
\begin{bmatrix}
\cos \theta&\sin\theta\\
-\sin \theta&\cos\theta
\end{bmatrix}
\\&=
\begin{bmatrix}
\cos \theta&-\sin\theta\\
\sin \theta&\cos\theta
\end{bmatrix}
\begin{bmatrix}
s_x\cos \theta&s_x\sin\theta\\
-s_y\sin \theta&s_y\cos\theta
\end{bmatrix}
\\&=
\begin{bmatrix}
s_x\cos^2 \theta+s_y\sin^2\theta&(s_x-s_y)\sin\theta\cos\theta\\
(s_x-s_y)\sin\theta\cos\theta&s_x\sin^2 \theta+s_y\cos^2\theta
\end{bmatrix}
}{}

If a point $\begin{bmatrix}p_x&p_y\end{bmatrix}^T$ is regarded as a fixed point, a translation is needed as much as
\eqn{
(I-R)\begin{bmatrix}p_x\\p_y\end{bmatrix}=
\begin{bmatrix}
1-s_x\cos^2 \theta-s_y\sin^2\theta&(s_y-s_x)\sin\theta\cos\theta\\
(s_y-s_x)\sin\theta\cos\theta&1-s_x\sin^2 \theta-s_y\cos^2\theta
\end{bmatrix}
\begin{bmatrix}p_x\\p_y\end{bmatrix}
}{}
















\end{document}